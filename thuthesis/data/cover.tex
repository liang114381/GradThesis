\thusetup{
  %******************************
  % 注意:
  %   1. 配置里面不要出现空行
  %   2. 不需要的配置信息可以删除
  %******************************
  %
  %=====
  % 秘级
  %=====
  secretlevel={秘密},
  secretyear={10},
  %
  %=========
  % 中文信息
  %=========
  ctitle={基于多方安全计算的隐私保护的药物分子相似性查询系统},
  cdegree={工学学士},
  cdepartment={交叉信息院},
  cmajor={计算机科学与技术(实验班)},
  cauthor={梁桢枭},
  csupervisor={曾坚阳\ \ 助理教授},
  % cassosupervisor={陈文光教授}, % 副指导老师
  % ccosupervisor={某某某教授}, % 联合指导老师
  % 日期自动使用当前时间,若需指定按如下方式修改:
  % cdate={超新星纪元},
  %
  % 博士后专有部分
  cfirstdiscipline={计算机科学与技术},
  cseconddiscipline={系统结构},
  postdoctordate={2009年7月——2011年7月},
  id={编号}, % 可以留空: id={},
  udc={UDC}, % 可以留空
  catalognumber={分类号}, % 可以留空
  %
  %=========
  % 英文信息
  %=========
  etitle={An Introduction to \LaTeX{} Thesis Template of Tsinghua University v\version},
  % 这块比较复杂,需要分情况讨论:
  % 1. 学术型硕士
  %    edegree:必须为Master of Arts或Master of Science(注意大小写)
  %             “哲学、文学、历史学、法学、教育学、艺术学门类,公共管理学科
  %              填写Master of Arts,其它填写Master of Science”
  %    emajor:“获得一级学科授权的学科填写一级学科名称,其它填写二级学科名称”
  % 2. 专业型硕士
  %    edegree:“填写专业学位英文名称全称”
  %    emajor:“工程硕士填写工程领域,其它专业学位不填写此项”
  % 3. 学术型博士
  %    edegree:Doctor of Philosophy(注意大小写)
  %    emajor:“获得一级学科授权的学科填写一级学科名称,其它填写二级学科名称”
  % 4. 专业型博士
  %    edegree:“填写专业学位英文名称全称”
  %    emajor:不填写此项
  edegree={Bachelor of Engineering},
  emajor={Computer Science and Technology},
  eauthor={Zhenxiao Liang},
  esupervisor={Professor Jianyang Zeng},
  % eassosupervisor={Chen Wenguang},
  % 日期自动生成,若需指定按如下方式修改:
  % edate={December, 2005}
  %
  % 关键词用“英文逗号”分割
  ckeywords={机器学习, 多方安全计算, 计算生物学, 奇异值分解},
  ekeywords={Machine Learning, Secure Multi-party Computation, Computational Biology, SVD}
}

% 定义中英文摘要和关键字
\begin{cabstract}
	药物分子的相似性计算是一个药物研发中进行信息共享时的一个基本问题。对于如何对已知的若干种药物分子进行相似性比较已经有大量成熟的方案,其中就包括分子指纹的方案。即先提取药物分子的分子指纹,然后对分子指纹进行比对。但这要求计算方持有待比较分子的所有分子指纹。这个要求对于很多制药公司的需求而言是不实际的。例如对于尚处于制药公司研发序列中的药品,制药公司可能不能接受公开整个研发中药物指纹,因为这会泄露过多的药物相关的信息给竞争者。然而,计算药物相似度的需求始终存在,因为制药公司总是希望能够对当前市场上的研发现状和其他公司研发中药物与自己公司药物的相似性有所了解。
	
	对此,本文实现了一种基于安全多方计算药物分子相似度比较系统。该系统通过将计算中间数据分布式地共享在多个不同的参与者或是服务提供商之间,可以在不泄露分子指纹信息的前提下得到药物分子间的相似性。同时,由于多方安全计算协议中比较运算符的低效性,为了提高系统运行的效率,我们实现了一种替代方案,即通过奇异值分解对原始数据进行降维,然后直接计算降维后数据的欧式距离来得到所需的分子间相似性。
\end{cabstract}

% 如果习惯关键字跟在摘要文字后面,可以用直接命令来设置,如下:
% \ckeywords{\TeX, \LaTeX, CJK, 模板, 论文}

\begin{eabstract}
	The computation for similarity between drugs is a basic problem when sharing information for drug development. There have been quite a few developed solutions for the cases where the structures of drug molecules are available, including the molecular fingerprint solution. Specifically, the fingerprint of some drug molecule, which is usually a hashed feature vector, would be calculated first, and then we can make comparison on the fingerprints of different drug molecules. However, such a condition that the pharmaceutical company must provide the full information of the drug structure is impractical. For example, the pharmaceutical company may not accept to publish such information about the drugs under development. But the demand for calculating similarity between all drugs still exists, because pharmaceutical company always hope to obtain some knowledge of the current situation of drug development.
	
	Thus, we implement an MPC-based system used to calculate the similarity between drug molecules. This system works by distributing intermediate data to multiple participants of service providers to obtain similarity information without leak the full information of drugs. Moreover, due to the inefficiency of comparison operator in MPC, we implement an alternative solution, which first uses the singular value decomposition to reduce the dimension of raw data and then calculates the Euclidean distance between reduced data to obtain similarities.
\end{eabstract}

% \ekeywords{\TeX, \LaTeX, CJK, template, thesis}
