\documentclass[degree=bachelor, tocarialchapter]{thuthesis}
% 选项
%   degree=[bachelor|master|doctor|postdoctor], % 必选,学位类型
%   secret,                % 可选(默认:关闭),是否有密级
%   tocarialchapter,       % 可选(默认:关闭),章目录中使用黑体(这项表示同时打开下面两项)
%   tocarialchapterentry,  % 可选(默认:关闭),单独控制章标题在目录中使用黑体
%   tocarialchapterpage,   % 可选(默认:关闭),单独控制章页码在目录中使用黑体
%   pifootnote,            % 可选(默认:关闭),页脚编号采用 pifont 字体符号,建议打开

% 所有其它可能用到的包都统一放到这里了,可以根据自己的实际添加或者删除。
\usepackage{thuthesis}

% 定义所有的图片文件在 figures 子目录下
\graphicspath{{figures/}}

% 可以在这里修改配置文件中的定义。导言区可以使用中文。
% \def\myname{薛瑞尼}

\begin{document}

%%% 封面部分
\frontmatter
\thusetup{
  %******************************
  % 注意:
  %   1. 配置里面不要出现空行
  %   2. 不需要的配置信息可以删除
  %******************************
  %
  %=====
  % 秘级
  %=====
  secretlevel={秘密},
  secretyear={10},
  %
  %=========
  % 中文信息
  %=========
  ctitle={清华大学学位论文 \LaTeX\ 模板\\使用示例文档 v\version},
  cdegree={工学学士},
  cdepartment={交叉信息院},
  cmajor={计算机科学与技术(实验班)},
  cauthor={梁桢枭},
  csupervisor={曾坚阳\ \ 助理教授},
  % cassosupervisor={陈文光教授}, % 副指导老师
  % ccosupervisor={某某某教授}, % 联合指导老师
  % 日期自动使用当前时间,若需指定按如下方式修改:
  % cdate={超新星纪元},
  %
  % 博士后专有部分
  cfirstdiscipline={计算机科学与技术},
  cseconddiscipline={系统结构},
  postdoctordate={2009年7月——2011年7月},
  id={编号}, % 可以留空: id={},
  udc={UDC}, % 可以留空
  catalognumber={分类号}, % 可以留空
  %
  %=========
  % 英文信息
  %=========
  etitle={An Introduction to \LaTeX{} Thesis Template of Tsinghua University v\version},
  % 这块比较复杂,需要分情况讨论:
  % 1. 学术型硕士
  %    edegree:必须为Master of Arts或Master of Science(注意大小写)
  %             “哲学、文学、历史学、法学、教育学、艺术学门类,公共管理学科
  %              填写Master of Arts,其它填写Master of Science”
  %    emajor:“获得一级学科授权的学科填写一级学科名称,其它填写二级学科名称”
  % 2. 专业型硕士
  %    edegree:“填写专业学位英文名称全称”
  %    emajor:“工程硕士填写工程领域,其它专业学位不填写此项”
  % 3. 学术型博士
  %    edegree:Doctor of Philosophy(注意大小写)
  %    emajor:“获得一级学科授权的学科填写一级学科名称,其它填写二级学科名称”
  % 4. 专业型博士
  %    edegree:“填写专业学位英文名称全称”
  %    emajor:不填写此项
  edegree={Bachelor of Engineering},
  emajor={Computer Science and Technology},
  eauthor={Zhenxiao Liang},
  esupervisor={Professor Jianyang Zeng},
  % eassosupervisor={Chen Wenguang},
  % 日期自动生成,若需指定按如下方式修改:
  % edate={December, 2005}
  %
  % 关键词用“英文逗号”分割
  ckeywords={机器学习, 多方安全计算, 计算生物学},
  ekeywords={Machine Learning, Secure Multi-party Computation, Computational Biology}
}

% 定义中英文摘要和关键字
\begin{cabstract}
  论文的摘要是对论文研究内容和成果的高度概括。摘要应对论文所研究的问题及其研究目
  的进行描述,对研究方法和过程进行简单介绍,对研究成果和所得结论进行概括。摘要应
  具有独立性和自明性,其内容应包含与论文全文同等量的主要信息。使读者即使不阅读全
  文,通过摘要就能了解论文的总体内容和主要成果。

  论文摘要的书写应力求精确、简明。切忌写成对论文书写内容进行提要的形式,尤其要避
  免“第 1 章……;第 2 章……;……”这种或类似的陈述方式。

  本文介绍清华大学论文模板 \thuthesis{} 的使用方法。本模板符合学校的本科、硕士、
  博士论文格式要求。

  本文的创新点主要有:
  \begin{itemize}
    \item 用例子来解释模板的使用方法;
    \item 用废话来填充无关紧要的部分;
    \item 一边学习摸索一边编写新代码。
  \end{itemize}

  关键词是为了文献标引工作、用以表示全文主要内容信息的单词或术语。关键词不超过 5
  个,每个关键词中间用分号分隔。(模板作者注:关键词分隔符不用考虑,模板会自动处
  理。英文关键词同理。)
\end{cabstract}

% 如果习惯关键字跟在摘要文字后面,可以用直接命令来设置,如下:
% \ckeywords{\TeX, \LaTeX, CJK, 模板, 论文}

\begin{eabstract}
   An abstract of a dissertation is a summary and extraction of research work
   and contributions. Included in an abstract should be description of research
   topic and research objective, brief introduction to methodology and research
   process, and summarization of conclusion and contributions of the
   research. An abstract should be characterized by independence and clarity and
   carry identical information with the dissertation. It should be such that the
   general idea and major contributions of the dissertation are conveyed without
   reading the dissertation.

   An abstract should be concise and to the point. It is a misunderstanding to
   make an abstract an outline of the dissertation and words ``the first
   chapter'', ``the second chapter'' and the like should be avoided in the
   abstract.

   Key words are terms used in a dissertation for indexing, reflecting core
   information of the dissertation. An abstract may contain a maximum of 5 key
   words, with semi-colons used in between to separate one another.
\end{eabstract}

% \ekeywords{\TeX, \LaTeX, CJK, template, thesis}

% 如果使用授权说明扫描页,将可选参数中指定为扫描得到的 PDF 文件名,例如:
% \makecover[scan-auth.pdf]
\makecover

%% 目录
\tableofcontents

%% 符号对照表
\begin{denotation}[3cm]
\item[HPC] 高性能计算 (High Performance Computing)
\item[cluster] 集群
\item[Itanium] 安腾
\item[SMP] 对称多处理
\item[API] 应用程序编程接口
\item[PI] 聚酰亚胺
\item[MPI] 聚酰亚胺模型化合物,N-苯基邻苯酰亚胺
\item[PBI] 聚苯并咪唑
\item[MPBI] 聚苯并咪唑模型化合物,N-苯基苯并咪唑
\item[PY] 聚吡咙
\item[PMDA-BDA]	均苯四酸二酐与联苯四胺合成的聚吡咙薄膜
\item[$\Delta G$] 活化自由能 (Activation Free Energy)
\item[$\chi$] 传输系数 (Transmission Coefficient)
\item[$E$] 能量
\item[$m$] 质量
\item[$c$] 光速
\item[$P$] 概率
\item[$T$] 时间
\item[$v$] 速度
\item[劝学] 君子曰:学不可以已。青,取之于蓝,而青于蓝;冰,水为之,而寒于水。木
  直中绳。輮以为轮,其曲中规。虽有槁暴,不复挺者,輮使之然也。故木受绳则直,金就
  砺则利,君子博学而日参省乎己,则知明而行无过矣。吾尝终日而思矣,不如须臾之所学
  也;吾尝跂而望矣,不如登高之博见也。登高而招,臂非加长也,而见者远;顺风而呼,
  声非加疾也,而闻者彰。假舆马者,非利足也,而致千里;假舟楫者,非能水也,而绝江
  河,君子生非异也,善假于物也。积土成山,风雨兴焉;积水成渊,蛟龙生焉;积善成德,
  而神明自得,圣心备焉。故不积跬步,无以至千里;不积小流,无以成江海。骐骥一跃,
  不能十步;驽马十驾,功在不舍。锲而舍之,朽木不折;锲而不舍,金石可镂。蚓无爪牙
  之利,筋骨之强,上食埃土,下饮黄泉,用心一也。蟹六跪而二螯,非蛇鳝之穴无可寄托
  者,用心躁也。—— 荀况
\end{denotation}



%%% 正文部分
\mainmatter
\chapter{引言}
\label{cha:intro}

随着生物技术及药品研发的发展,制药企业的药品实验的成本和规模也日渐增加。如果各个企业仅是独立地进行药物研发,无疑会产生大量的资源浪费。例如当某个企业已经实验过某种药物的特性,但尚未公开,则另外一个企业对于同种或类似药物但实验则意味着资源的浪费。对此,企业之间对正在研发药物的信息共享可以很好的避免这一类问题。不幸的是,由于专利和对盗用的担忧,绝大部分企业不愿意如实地向竞争对手公开研发中的药物信息。

一种可能的解决方案是通过可信任的第三方来实现信息共享。但这至少可能具有以下问题:
\begin{itemize}
	\item 需要找到一个各方都能够信任的第三方,但这在很多情况下并不现实或者完美。例如机构很有可能对于参与共享的制药企业有事实上的利益偏好(如来自不同的国家或是地区的企业),从而有泄漏信息的动机以及由此带来的风险。
	\item 当整个系统足够庞大时,第三方本身的安全性仍然需要更多的措施来保障。一个集中各方数据的第三方很自然的会是攻击者的首要目标。第三方数据库的泄漏将将远大于两个药企之间直接共享数据带来的损失。
\end{itemize}

对此,我们的想法是,一个去中心化,或是部分去中心化的解决方案可以很大程度上地解决如上这些问题。具体地说,我们考虑采用MPC(Secure Multi-party Computation,多方安全计算)来实现制药企业之间所需的信息共享。由于采用MPC协议可能需要交换的信息量过大,这时我们考虑先进行一次 tf-idf 采样,然后用SVD算法将数据点(本文中通常是药物分子的指纹)投影到低维空间\citeu{Wan086033}以降低需要交互数据量。下面我们将对所涉及的数个基本概念进行简单介绍。

\section{分子的相似度及分子指纹的基本概念\citeu{CERETOMASSAGUE201558}}

不同分子之间的相似度本身是一个相对主观的概念。分子结构的复杂性也导致了一般意义上这个问题的困难。对此,通常的做法是对原始的分子结构进行一定程度的抽象和简化。其中,最常见的方式是将分子结构转化成一个比特串,即所谓的分子指纹(Molecular Fingerprint)。

假设我们得到了两个分子的指纹,分别是比特串$A[1\dots l]$和$B[1\dots l]$。这里为简单其见,我们假设这个指纹具有定长$l$。这样一来,我们可以在$A,B$间定义一些常见的分子相似度如下:
\begin{table}[H]
	\centering
	\caption{一些常见的相似度定义\citeu{CERETOMASSAGUE201558}}
	\begin{tabular}{l l l}
		\hline
		相似度名称 & 表达式 & 取值范围\\
		\hline
		Tanimoto系数 & $\frac{c}{a+b-c}$ & $[0,1]$\\
		欧氏距离 & $\sqrt{a+b-2c}$ & $[0,\sqrt{l}]$\\
		汉明距离 & $a+b-2c$ & $[0,l]$\\
		Dice系数 & $\frac{2c}{a+b}$ & $[0,1]$\\
		余弦相似度 & $\frac{c}{\sqrt{ab}}$ & $[0,1]$\\
		\hline
	\end{tabular}
\end{table}
其中
\begin{align*}
a&=|\{i:A[i]=1\wedge B[i]=0\}|\\
b&=|\{i:A[i]=0\wedge B[i]=1\}|\\
c&=|\{i:A[i]=1\wedge B[i]=1\}|
\end{align*}

下面我们考虑如何去计算分子指纹的问题\citeu{CERETOMASSAGUE201558}。大多数分子指纹只用到了二维的分子图,而少部分可能用到分子的三维信息,后者在药效团指纹中尤其常见。主要的指纹采集方法包括基于键值的子结构指纹,拓扑或基于路径的指纹,或者是圆形指纹。

基于键值的子结构指纹中的某一位将反应其代表的分子是否包含某个分子子结构(如氨基,羟基等)。当分子结构可以被指纹中所包含的若干子结构所表示时,这种指纹将比较有用,反之则不能很好地表示原分子结构。

而拓扑或基于路径的指纹将遍历分子中所有可能的(一定长度的)路径$p$,并对路径进行哈希得到哈希值$h_p$,然后将指纹中对应位置$h_p$处赋为1。这种指纹可以用于快速的分子子结构筛查及搜索。

类似于拓扑指纹,圆形指纹也利用了分子结构的哈希。但不同于拓扑指纹中考虑可能的路径,圆形指纹中考虑对以每个原子为中心一定半径圆内的分子子结构进行哈希来得到最终的分子指纹。这种指纹不适用于子结构查询(因为相同的子结构碎片可能对应不同的局部环境),但其被广泛应用于分子的整体结构相似性搜索。本文实验部分将主要使用某一种圆形指纹来实现相似度比对操作。

\section{MPC的基本概念\citeu{intro_to_mpc}}
\label{section: mpc}
MPC(Secure Multi-party Computation,多方安全计算)被用于包含$n$个参与者的计算问题。我们希望在不泄漏中间信息的前提下得到想要多方计算的函数。更具体地说,假设我们有$n$个输入$x_1,\dots,x_n$,其中第$i$个参与者知道$x_i$但不知道别的输入,我们希望计算多元函数$f$合于
\begin{equation}
f(x_1,\dots,x_n)=(y_1,\dots,y_n).
\end{equation}
并满足第$i$个参与者只能得到$y_i$但得不到其它任何信息。

我们这里介绍一种Shamir协议来实现这个目标。为此我们先引入秘密共享(secret sharing)的概念。首先定义$t=\lfloor(n-1)/2\rfloor$。

假设参与者$i$持有数字$x$(事实上这里的$x\in\mathbb{Z}_p$而不是任意整数),那么参与者$i$可以(均匀)随机地选取一个$\mathbb{Z}_p$上的多项式$g$合于$\deg(g)\leq t$且$g(0)=x$。然后,参与者$i$将$g(j)$发送给参与者$j,\ j=1,\dots,n$。这样一个整个操作称为参与者$i$进行了数字$x$的秘密共享。

注意到由拉格朗日插值法,只要有超过$t$名参与者选择共享他们所持有的共享(如$g(1),\dots,g(t+1)$),那么我们可以重建出$g$进而恢复出所需的$x=g(0)$。另一方面,如果我们只已知$g$在$t$个点上的取值,那么$t$次多项式$g$在所得的条件概率空间中依然满足$g(0)$是$\mathbb{Z}_p$上的均匀分布,即没有任何信息能够从中恢复。因此这保证了联合的攻击者不超过$t$个时$x$的安全性。我们记$[x]$为$x$的共享的集合,而$x_i$表示分配给第$i$个参与者的共享。

以秘密共享机制为基础,我们可以实现(秘密的)数字上的算术运算。由于秘密共享机制是线性的,加法可以直接通过所有持有共享的参与者在本地对他们所持有的共享来完成。例如$[a]+[b]$即可得到$a+b$在所有参与者上的秘密共享。

对于乘法$a\times b$的相对要复杂一些。假设$f,g$是潜在的分别对应于$a,b$的$t$次多项式,且$h=fg,\deg(h)=2t$。首先需要注意到$a_ib_i$对应于多项式$fg$在这$n$个参与者上的秘密分配,但$fg$是一个$2t$次多项式,不满足我们的要求。因此我们这里应用一个技巧,即对每个$i$,参与者$i$计算$a_ib_i$对应的共享$[a_ib_i]$然后分配给所有$n$个参与者。由拉格朗日插值及$n>2t=\deg(h)$可知,存在常系数$\lambda_1,\dots,\lambda_n$使得
\begin{equation}
h(0)=\lambda_1 h(1)+\dots+\lambda_n h(n).
\end{equation}
从而我们可以(在每个参与者本地)计算
\begin{align}
[c] &:= \lambda_1[a_1b_1]+\dots+\lambda_n[a_nb_n]\nonumber\\
&=[\lambda_1a_1b_1+\dots+\lambda_na_nb_n]\nonumber\\
&=[\lambda_1h(1)+\dots+\lambda_n h(n)]\nonumber\\
&=[h(0)]=[ab]
\end{align}
这样我们就得到了$ab$所对应的$n$个共享而没有泄露任何关于$a,b$的信息。

\section{TF-IDF统计量的基本概念\citeu{Rajaraman:2011:MMD:2124405}}
\label{section: tfidf}

在信息检索领域,TF-IDF(term frequency-inverse document frequency)是一种统计方法以评估一个单词对于一个语料库中的某个文件的重要性。直觉上来说,显然一个单词对某个文件的重要性与在该文件中出现的次数成正比,但同时随着它在语料库中出现总频率的增加而下降。因此我们将tf-idf统计量定义为tf乘以idf。tf-idf统计量常被用于搜索领域以确定关键词与对应文档(页面)的关联度。TF(词频率)和IDF(文档频率倒数)的几种常见的定义如下

\begin{table}[H]
	\centering
	\begin{subtable}{0.45\textwidth}
		\centering
		\caption{几种TF的常见定义}
		\begin{tabular}{ll}
		二元定义 & 0,1\\
		原始计数 & $f_{t,d}$\\
		词频率 & $f_{t,d}/\sum_{t'\in d}f_{t',d}$\\
		对数正规化 & $\log(1+f_{t,d})$
		\end{tabular}
	\end{subtable}
	\begin{subtable}{0.45\textwidth}
		\centering
		\caption{几种IDF的常见定义}
		\begin{tabular}{ll}
		平凡定义 & 1\\
		文档频率倒数 & $\log\frac{N}{n_t}=-\log\frac{n_t}{N}$\\
		平滑后的文档频率倒数 & $\log\left(1+\frac{N}{n_t}\right)$
		\end{tabular}
	\end{subtable}
	\caption{TF及IDF的常见定义}
\end{table}
其中$f_{t,d}$表示单词$t$在文档$d$中出现的频率,$n_t$表示有单词$t$出现的文档的数量,$N$是总的文档数量。对于某个单词$t$,文档$d$及语料库$D$,$t$对于$d$的tfidf统计量将被定义为
\begin{equation}
TFIDF(t,d,D)=tf(t,d)\times idf(d,D).
\end{equation}

\section{SVD算法的基本概念}

SVD(Singular Value Decomposition,奇异值分解)算法被用于将一个$m\times n$的矩阵$M$分解为
\begin{equation}
M=U\Sigma V^T,
\end{equation}
其中$U$和$V$分别是$m\times m$和$n\times n$的酉矩阵。而$\Sigma=\textup{diag}(\sigma_1,\dots,\sigma_r,0,\dots,0)$是一个前$r$项非零的对角线矩阵,满足$\sigma_1>\dots>\sigma_r>0$,这里$r$是矩阵$M$的秩。标准的SVD算法时间复杂度为$O(\min\{mn^2,m^2n\})$。

SVD算法可以自然地用于将高维空间的点集投影到低维空间中去。特别地,我们假设$M$的每一列代表一个$\mathbb{R}^m$空间中的一个点,于是$M$表示一个包含$n$个点的$\mathbb{R}^m$中的点集。现在我们希望找到一个矩阵$H\in\mathbb{R}^{d\times n}$用来表示这$n$个点被投影到$\mathbb{R}^{d}$中的结果。我们希望找到一个$\mathbb{R}^m$中的$d$维超平面,让原始点到这个超平面上的投影点距离的平方和最小。可以证明\citeu{Stewart92onthe},
\begin{equation}
M^*=U_d\Sigma_dV_d^T
\end{equation}
即是原始点在该超平面(在$\mathbb{R}^m$中)的投影坐标,其中$U_d$表示$U$的前$d$列,$V_d$表示$V$的前$d$列,$\Sigma_d=\textup{diag}(\sigma_1,\dots,\sigma_d)$。对这$n$个点进行$\mathbb{R}^n$上的旋转变换$U^T$得到
\begin{equation}
H^*=U^TM^*=\begin{pmatrix}
I_r\\ O_{m-r}
\end{pmatrix}
\Sigma_d V_d^T.
\end{equation}
容易看到
\begin{equation}
H=\Sigma_d V_d^T
\end{equation}
即是我们需要的在超平面上的投影的坐标。

\section{本文的主要工作}

首先,我们使用python给出了一个基于Shamir共享协议的MPC实现。主要包括了秘密分配,秘密恢复,安全加法,安全乘法及安全二进制比较等功能的实现。在整体框架上使用多线程及消息队列以对多台服务器及客户端进行模拟,客户端将使用非对称加密保密地提交本地数据至服务器,以减少MPC协议中参与者的数量,提高反应速度。此外,在服务器实现上加入了所需的同步化以保证模拟的正确运行。

我们在合成数据(向量)上及真实的药物指纹数据上进行了MPC比对的相关实验。如比对若干向量或是指纹的相似性并得到最接近的那个向量。药物信息主要来源于公开的药品数据库DrugBank。鉴于指纹向量的稀疏性,在指纹数据的实验中使用了两种不同的实现方式:
	
第一种是通过对向量上的非零项的坐标进行扫描式的比对以得到所需结果。这种方法最为直接,但这会引入大量的比较运算。在通常的本地运算中这不会是一个问题,但MPC的比较运算实现相当缓慢,因此造成了效率的相对低下。

第二种方案中,我们将先从原始的指纹数据中提取特征向量,然后直接计算特征向量的距离(如欧氏距离)。这样的距离计算由于只涉及基本运算,其效率会比比较运算的实现快很多。但值得注意的是,特征向量的元素的值域很可能是实数而不是整数,由于我们的MPC算法运行在有限域上,因此我们需要对输入做必要的离散化。
	
关于特征向量,我们将先计算tf-idf矩阵然后对矩阵进行SVD降维以得到对应的特征向量。tf-idf的计算可以简单地通过MPC中的加法来完成,而对较为复杂的SVD,我们选择先通过本地计算本地数据的SVD,然后再在服务器上进行SVD结果的合成和迭代以得到近似解。如果原始的全体数据在各个服务器上的部分具有统计上的一致性,则可以预期这种做法能够取得较为准确的全局SVD解。

我们在合成数据及真实数据上比对了两种方案的效率和结果。需要注意的是第二种方案的效率很大程度上取决于离散化的精度及分布式SVD算法的选取,且两种方案得到的查询结果也不一定一致。而结果表明采用合适的参数的情况下,第二种方案在保证查询准确性的前提下相比方案一具有更高的效率,这里准确性可以通过它们是否具有相似的实验性质来表示(如是否和某个蛋白质反应,是否具有某种毒性等)
\chapter{相关工作}


%%% 其它部分
\backmatter

%% 本科生要这几个索引,研究生不要。选择性留下。
% 插图索引
\listoffigures
% 表格索引
\listoftables
% 公式索引
\listofequations


%% 参考文献
% 注意:至少需要引用一篇参考文献,否则下面两行可能引起编译错误。
% 如果不需要参考文献,请将下面两行删除或注释掉。
% 数字式引用
\bibliographystyle{thuthesis-numeric}
% 作者-年份式引用
% \bibliographystyle{thuthesis-author-year}
\bibliography{ref/refs}


%% 致谢
% 如果使用声明扫描页,将可选参数指定为扫描后的 PDF 文件名,例如:
% \begin{acknowledgement}[scan-statement.pdf]
\begin{acknowledgement}
  衷心感谢导师 xxx 教授和物理系 xxx 副教授对本人的精心指导。他们的言传身教将使
  我终生受益。

  在美国麻省理工学院化学系进行九个月的合作研究期间,承蒙 xxx 教授热心指导与帮助,不
  胜感激。感谢 xx 实验室主任 xx 教授,以及实验室全体老师和同学们的热情帮助和支
  持!本课题承蒙国家自然科学基金资助,特此致谢。

  感谢 \LaTeX 和 \thuthesis\cite{thuthesis},帮我节省了不少时间。
\end{acknowledgement}


%% 附录
\begin{appendix}
\chapter{外文资料的书面翻译}

\title{Secure Multiparty Computation Goes Live}

{\heiti 摘要:} 2008年1月我们进行了多方安全计算的首次大范围的实际应用,本文就此次应用及其中所采用的密码学方法进行报告。

\section{引言及历史}
在多方计算(MPC)中,我们考虑了一系列参与者 $P_1,\dots,P_n$. 每个人
一开始持有输入 $x_1,\dots,x_n$,我们想在这些输入上安全地计算一个函数 $f$ 满足$f(x_1,\dots,x_n)=(y_1,\dots,y_n)$,使得$P_i$可以得到$y_i$但无法获得其他信息。我们允许参与者表现出一定程度的敌对行为而这个性质依然成立。这个目标可以通过让所有参与者遵从某个协议$\pi$进行交互来实现。直觉上来说,我们希望执行$\pi$等价于有一个被信任方$T$先从$P_i$私下接收$x_i$,然后计算函数$f$,并最后将$y_i$返回给每个$P_i$。有了这样的协议,我们原则上可以解决几乎所有的密码协议问题。MPC的一般理论始于80年代后期[16,3,7]。这个理论后来被发展出有几种实现方式(例如[21,18,8])。可以在[6]中找到已知理论结果的概述。

尽管MPC在解决许多的问题方面具有明显的潜力,但我们可以看到在过去几乎没有MPC的实际应用。这可能部分是由于以下事实:直接实施原始的通用MPC协议将导致解决方案非常低效。另一个因素则是广大公众长期对该技术潜力的普遍缺乏理解。事实上大量的研究已经用于解决效率问题,包括通用协议[11,17,9]以及诸如投票等特殊类型的计算[4,12]。

另一个不同的研究路线明确地关注一系列经济应用,这对于实际应用特别有帮助。例如,两个本文作者参与的项目就采取了这种方法:SCET(Secure Computing, Economy and Trust)2和SIMAP(Secure Information Management and Processing)。它们在本文中被用于描述MPC的实际应用。在经济领域的机制设计中,一个值得信赖的第三方的概念自70年代以来一直是一个重要的假设[15,19,10]。自从该领域诞生以来,它的发展势头迅猛,变成了一个真正的交叉学科。今天,许多实际的机制设计需要一个值得信赖的第三方是很自然的考虑。而MPC提供了实现这一点的一种可能性。特别地,我们考察过一下这些场景:
\begin{itemize}
	\item 由于各种原因需要进行密封投标的许多类型的拍卖。 其中最著名的可能是密封投标的标准最高出价拍卖。而另一个考虑成交额的常见拍卖类型是许多卖家和买家参与的所谓的双向拍卖。 这种拍卖用于解决寻到商品的公平市场价格和市场需求。
	\item 基准测试。多家公司想要整合它们的业务信息,以便将自己与该地区的最佳做法进行比较。基准测试过程被用于学习,计划或别的目的,而这个过程自然必须在保护公司数据私密性的前提下进行。
\end{itemize}

在考虑这样的应用时,人们发现所需的计算基本上是中等规模(如大约32位)整数上的初等算术。更具体地说,在相当广泛的情况中只会用到加法,乘法和整数比较。加法
和乘法可以通过著名的通用MPC协议相当有效地处理。实际上,因为协议基于$\mathbb{Z}_p$的秘密分享,该协议的做法是选择与输入数字相比足够大的$p$,然后进行模素数$p$的运算。我们可以避免模块化的减少和获得整数加法和乘法。

由于每个数字都是使用线性秘密共享方案来“一件一件”共享的,这种方案有时是高效的。以安全加法为例,每个玩家只需要进行一次本地加法即可。不幸的是,这也意味着比较运算要困难得多。通用的解决方案通过在$\mathbb{Z}_p$上进行算术电路运行来实现比较运算,但由于电路无法直接地访问输入的二进制表达,从而所需的算术电路规模会非常大,以至于其无法成为实际的解决方案。因此我们必须开发用于比较的特殊用途技术。这方面的一个例子是固定轮数的无条件安全的比较协议[13]。

\section{主要贡献}
在本文中,我们报道了使用MPC实施的第一次大规模实际实验 --- 一个安全的拍卖。我们阐述了其应用场景并论证了多方计算是该问题一个很好的解决方案。我们描述并报告了该系统的运作方式。 最后,我们详细阐述了所使用的密码协议并证明了它们的安全性。同时,对于实践中常见的数字规模,我们提出了一个相比[13]更实用的对数循环比较协议。

\section{应用场景}
在本节中,我们将描述我们系统部署的实际情况。 在[1]中已经初步描述了该场景的计划和小规模演示的结果。

在丹麦,有数千农民生产甜菜,并销售给丹麦市场上唯一的甜菜加工商Danisco。农民们会签署一份合同,该合同规定了他们会向Danisco提供一定数量的甜菜,而丹尼斯克会根据某种定价方案向他们支付一定费用。 这些合同可以在农民间交易,但这样的交易历来非常有限,且主要通过双边协商完成。

但近年来,欧盟大幅减少了对甜菜生产的支持。再加上其他因素,这意味着现在迫切需要重新分配合同给农民以使他们得到最佳的回报。人们意识到,这最好是通过全国范围内的交换完成,即双向拍卖。

{\heiti 市场清算价格}:特定商业案例的详细信息可以在[2]中找到。这里,我们简要总结一下重点,稍后将给出所要进行的实际计算的更多细节。我们的目标是找到所谓的市场清算价格(MCP),即每单位待交易的价格。具体来说,每个买方将给出对于每个可能的价格,他愿意用该价格购买多少数量的商品。同样的,卖家将给出他们愿意在每个可能的价格上卖出多少商品.所有的出价都将交给拍卖商。他们会计算出每种价格下的总供求市场。由于我们可以假设随着价格增加供给增加而需求减少,我们可以找到一个价格使得总的供给等于总的需求,拍卖商将把这个价格作为清算价格。最后,所有的参与者将依照这个清算价格进行交易。

{\heiti 使用安全MPC进行投标的保密性}:一个满意的这类拍卖的实现需要考虑到安全问题:投标显然会泄露个人信息,例如农民的经济地位和他的生产力,因此鉴于Danisco在市场上的地位,农民不愿意接受Danisco作为拍卖商。即使Danisco绝不会在正在进行的合同(包括定价方案)再谈判中滥用其所获得的拍卖信息,仅仅对会发生这种情况的担忧可能会影响农民的出价并导致拍卖效果不理想。另一方面,一份合同中的标称数量由Danisco管理(并经常根据欧盟的管理调整)。在某些情况下,合同充当农民对于Danisco债务的担保。因此,独立于Danisco运行拍卖也是不现实的。最后,通过委托法律责任和实际责任给被信任的拍卖人,比如顾问机构,来完成拍卖将会非常昂贵。

最终决定的解决方案是进行电子双重拍卖。这里拍卖人实际上是来自Danisco,DKS和SIMAP项目的代表进行的三方计算。这个三方解决方案被选中部分原因在于它很自然地适合于这个场景,同时也因为它允许使用高效的信息学理论工具比如秘密共享,而不是更昂贵的加密方法,如同态加密。

{\heiti 动机}:是什么动机使得DKS和Danisco尝试使用这样一个新的、未试验的技术呢?一个重要因素是全国性的生产权利交换的明显需求在一前是不存在的,所以无论安全与否,低成本的电子化解决方案是一个巨大优势。但是,我们确实相信安全性也很重要。一个与拍卖有关的在线调查显示,农民确实会关心他们出价的保密性(见图1和图2中的表格)。如果Danisco和DKS试图用传统的方法进行拍卖,一个或多个人必然会得到这些出价,或以明文形式控制出价的系统。 结果是我们必须商定一些安全性政策,比如回答诸如以下问题:谁应该有权访问数据?什么时候有这个权限?如果数据泄漏,谁负责?怎么负责?

由于参与者们利益冲突,这将导致冗长的讨论,甚至可能让整个项目停摆。在对Danisco和DKS(甜菜种植者协会)有关决策者的采访中,
这些保密问题都得到了认可。如同前面提到的,雇用咨询机构作为协调人会昂贵的多,并且各方必须对协调人提出的安全政策是否令人满意达成共识。有趣的是,我们没有必要进行这种谈判,因为多方计算确保没有人会在任何时候获得出价信息。

值得注意的是,半诚实的模型中,安全多方计算已经实现了人们可以“选择不知道”。因此我们判断,一个半诚实安全的方案对于实现拍卖人的已经足够了。但请注意,来自投标者们的主动攻击仍然是可能的。虽然我们认为在我们的具体情况中,这不是一个重大风险,接下面我们仍会给出防范恶意投标者的协议。

有人可能会问,我们真的需要多方计算的全部威力吗?我们的方案确保没有任何一个单一玩家可以获得任何敏感信息,而且看上去在类似的信任模型中我们可以使用投票协议中的常用技巧来更有效率地解决这个问题:一个当事人$P_1$收到来自投标人的加密过的投标书。投标书是用另一方$P_2$的公钥加密的。然后$P_1$发送随机置换后的加密信息给$P_2$. $P_2$解密收到的消息并计算市场清算价格。因为$P_2$不知道谁放置了哪些出价,这种方案实现了一些安全性,但我们必须记住出价包含的信息比结果(市场清算价格)所传递的信息多得多。例如,可以查看人们愿意以清算价格以外的其他价格购买或出售的数量。原则上,这种类型的信息对于像Danisco这样的单一经营者来说是非常有价值的,以便发挥其市场能量。比如用于确定扩展或减少总的处理量的价格。这样的情况有多大程度会在实际生活中发生是很难回答的。我们的结论是使用全面的多方计算是一个更好的解决方案,因为它可以让我们根本不去考虑这个问题。

\section{加密协议}
抽象掉一些不太相关的细节后,我们的场景如下:一些输入客户端$I_1,\dots,I_m$为多方计算提供输入,这将由服务器$P_1,\dots,P_n$执行。在我们感兴趣的案例类型中,$m$非常大而$n$被认为是小的常数。 在我们的具体案例中,$n=3$而$m$约为1200。

来自客户端$I_i$的输入是非负整数的有序列表$\{x_{ij}|j=1,\dots,P\}$,其中下标$j$表示$x_{ij}$对应$P$个可能的单位价格中的一个,这$P$个价格按照增序排列。这样的名单被称为一个出价。 一个出价被称为出售出价,如果对应的列表是非减的,或者被称为买入出价如果对应的列表是非增的。对于一个买入出价,$x_{ij}$是投标人希望以第$j$个单位价格购买的数量。类似地我们定义$y_{ij}$为投标人愿意以价格$P_j$出售的数量。 由于实际条件的限制,这些输入必须能够以非交互方式(安全地)传送到服务器。

安全计算包括计算每个价格的总的需求和供给,即
$$
d_j=\sum_{i}x_{ij},\ s_j=\sum_i y_{ij},\ j=1,\dots,P,
$$
并最终找到合于$d_{j_0}-s_{j_0}=0$的$j_0$, 或者只是让这个差尽可能接近0(但可能不等于0)。由于可能的($P$个)价格是离散的,我们期望找到差精确地等于0的$j_0$。这也意味着必须要有一致认可的规则用来确定是接受供给略大于需求还是需求略大于供给。在我们的这个具体案例中我们制定了这样的规则,但其细节超出了这篇文章的范围。

任何情况下,随着价格的增加,供给增加而需求减少,因此我们能够通过二分查找很快找到所需的$j_0$. 注意到公开进行这样的比对是安全的:我们想要得到的$j_0$最终是公开的。因此对任意$j$,$d_j$和$s_j$之间的比较结果都可以从公开的$j_0$得到。最后除了$j_0$会被公开,所有对应于价格$j_0$的买入卖出量$x_{ij_0},y_{ij_0}$都会被公开。

因此我们只需设计一个协议实现理想功能$\mathcal{F}$:
\begin{enumerate}
	\item
	来自客户端$I_j$的输入$\textup{Input}(x_1,\dots,x_P)$:$x_1,\dots,x_P$是一个整数列其中每个数之多$l$位($l$是常数)长。此外这个数列要么是不减的,要么是不增的。$\mathcal{F}$将这些数字存储到唯一命名的寄存器中,并通知所有的参与者和对手一个输入数列已经从$I_j$处收到以及它们对应的寄存器名。
	\item
	输入 $C=A+B$:$A,B,C$ 是$\mathcal{F}$的寄存器名。$\mathcal{F}$计算$A$中数字和$B$中数字的和并将结果存储在$C$中。
	\item 
	输入 $C=A\times B$:$A,B,C$ 是$\mathcal{F}$的寄存器名。$\mathcal{F}$计算数字$A$中数字和$B$中数字的积并将结果存储在$C$中。
	\item
	输入 $\textup{ConstantMult}(a,B)$:$a\in \mathbb{Z}_p$,$B$是一个寄存器。$\mathcal{F}$计算$a$和$B$中数字的积并将结果存储在$B$中。
	\item
	输入 $\textup{Compare}(A,B)$, $\mathcal{F}$ 发送1给所有服务器如果$A$中的数字大于$B$中的数字,否则发送0给所有服务器。
	\item 
	输入 $\textup{Open}(A)$, $\mathcal{F}$ 发送寄存器$A$中所存数字到所有服务器。
	\item 
	输入 $\textup{RandomBit}(A)$, $\mathcal{F}$ 从$0/1$中随机选择一个数然后存储在寄存器$A$中。
\end{enumerate}

我们将假设一个静态且被动的对手,其可以破坏任意数量的输入客户端以及任意少数的服务器。我们接下来会说明以效率为代价,我们可以允许主动的,对客户端的攻击。然而在我们这个具体的案例中,我们认为这种来自客户端的主动攻击的风险非常小,以至于没有必要去付出效率上的代价。

我们的实现基于标准的Shamir秘密共享,包含$n$台服务器。我们用到了一个素数域$\mathbb{Z}_p$其中$p$被特别选取以满足其(二进制)长度恰好为$l+\kappa$。$\kappa$是一个用于控制比较协议的统计性质安全的参数。在我们这个具体案例中,$l=32$且$p$是一个65位素数。

令$t=\lfloor(n-1)/2\rfloor$,当需要秘密共享一个数字时,随机的选取一个至多$t$次的多项式$f$合于$f(0)=x$.这么一来$x$的$n$个共享将是$f(1),\dots,f(n)$。我们这里用$[x]$表示$x$的$n$个共享的集合。

设$\mathcal{F}'$是另一个功能,其除了不支持比较命令外,完全和$\mathcal{F}$一样。接下来我们会阐述如何实现$\mathcal{F}'$,以及如果基于$\mathcal{F}'$实现$\mathcal{F}$。

{\heiti 配置公钥:} 我们的实现假设公钥/私钥已经在计算开始前在服务器上被配置完成,且公钥对所有客户端公开。更精确地说,对于每个极大的不合格集合(即$|A|=t$),我们需要所有不在$A$中的服务器都持有私钥$sk_A$,且公钥$pk_A$对所有参与者(包括客户端和服务器)公开。

{\heiti 非交互式输入:} 第一个问题是现在如何去实现一个客户端输入数字$x_1,\dots,x_P$的命令。最直接的做法是秘密共享每个$x_i$然后用对应服务器的公钥进行加密。由于存在多个服务器,这样做会导致客户端需要发出多倍的信息。

因此我们给出一种非交互式VSSS技术[14]。简洁起见,在这里我们基于我们的具体案例($n=3$)来描述这种技术。首先我们有3个密钥对$(pk_i,sk_i),i=1,2,3$,服务器$i$持有两个私钥$sk_j,j\neq i$。现在令$f_i(x),i=1,2,3$表示3个次数不超过1的多项式,并满足$f_i(0)=1,f_i(i)=0$。现在可以按如下方式以加密形式传递一系列数字$x_1,\dots,x_P\in\mathbb{P}$给服务器组:
\begin{enumerate}
	\item 选择伪随机函数(PRF)$F$的三个密钥$K_1,K_2,K_3$。其以指数$j$为输入并输出一个$\mathbb{Z}_p$中的元素。
	\item 输出加密$E_{pk_i}(K_i),i=1,2,3$.
	\item 对于$j=1,\dots,P$,计算并输出:
		$$y_j=F_{K_1}(j)+F_{K_2}(j)+x_j\ \mod\ p$$
\end{enumerate}
每个服务器$P_a$现在可以处理这样一个加密并计算每个数字的Shamir共享:
\begin{enumerate}
	\item 从$E_{pk_i}(K_i)$中解密两个明文,这里$i\neq a$。
	\item 计算$x_j$的共享$share_{a,j}$:
	$$share_{a,j}=y_j-F_{K_1}(j)f_1(a)-F_{K_2}(j)f_2(a)-F_{K_3}(j)f_3(a)$$
	注意这里$f_a(a)=0$,所以我们可以不用知道$K_a$是多少。
\end{enumerate}

如果我们定义多项式$g_j$为$g_j=y_j-F_{K_1}(j)f_1-F_{K_2}(j)f_2-F_{K_3}(j)f_3$,那么
$$\deg(g)\leq 1,g_j(0)=x_j,g_j(a)=share_{a,j}.$$
因此我们已经得到一个有效的共享集。

我们可以使用阈值为$t$的Shamir共享,将这种做法推广到任意数量的服务器:首先进行一般化的密钥配置,即对每个大小为$t$的服务器,配置一个密钥对$(pk_A,sk_A)$,将$sk_A$发送给所有不在$A$中的服务器。然后找到$t$次的多项式$f_A$满足$f_A(0)=1$且$f_A(i)=0$对于所有$i\in A$。当然这种做法无法被用于较大的$n$,但在我们的应用中不需要考虑这一点。

这种方法有如下几个优点:
\begin{enumerate}
	\item 除了取决于服务器数量的额外的开销,被加密的数列和原数列有相同的规模。
	\item 假设公钥体系的解密算法是确定性的,那么解密过程总会产生某个数列的一致的共享。
	\item 如果某个服务器丢失了它的私钥,那么这个私钥可以从其它服务器处重新获得。
	\item 我们只需要从客户端到向服务器的通讯。这在实际使用中会非常方便,尤其是当我们只能控制相对较少数量的服务器的配置而不是相对较多的客户端配置时。比如说防火墙可能会使得服务器向客户端发送数据更加困难。
\end{enumerate}

{\heiti 加法和乘法:}在输入阶段后,所有的数字通过次数不超过$t$的多项式进行共享。我们可以通过标准协议来实现加法和乘法,并保持不变性质:$\mathbb{F}$中存储在寄存器中的数字和真实协议中秘密共享的数字保持一致。作用在输入$[a],[b]$上的加法运算可以通过令所有服务器本地执行加法运算来完成。显然$[a]+[b]=[a+b]$,因为秘密共享的算法时线性的。类似地,乘以一个常数可以通过所有服务器将它持有的共享呈上这个公共常数来完成。对于一般的乘法,服务器$P_i$先计算输入$a,b$的共享乘积$d_i=a_ib_i$,$P_i$向所有服务器分发共享$d_i$并得到$[d_i]$。最终,所有服务器计算$[ab]=\sum_i \lambda_i[d_i]$,其中$\lambda_i$是拉格朗日插值的系数。对任意不超过$n-1$的多项式$g$,其满足$g(0)=\sum_i \lambda_ig(i)$。由于$2t\leq n-1$,所以该插值可以用于重建$[ab]$。

{\heiti 随机比特:}我们用一个技巧[13]来实现函数RandomBit。所有的服务器秘密分别共享一个随机量,然后在本地对它们求和得到一个未知的随机数$[u]$。然后我们计算$[v]=[u^2\mod p]$并公开$v$。如果$v=0$那么重新再计算一次。否则继续公开地计算$v$的平方根$w$,并约定选择较小的那一个。接下来计算$w^{-1}[u]\mod p$,这个数会以$1/2$的概率为1以及$1/2$的概率为-1。因此$[(w^{-1}u+1)2^{-1}\mod p]$将给出我们需要的随机比特。

\textbf{引理1.} 如果所使用加密是语义安全的且$PRF$也是安全的,那么上述的协议在如下意义下安全地实现了$\mathbb{F}'$:攻击的对手是静态的、被动的,其可以破坏任意数量的客户端以及至多$t$台服务器。

证明略。

{\heiti 无需信任客户端的输入:} 上述方法无论客户端做什么都能产生一致的共享数字,但原则上客户端可能发送很大的数字,这可能会导致计算失败。我们可以防止这种情况发生,即我们将伪随机值$F_{K_i}(j)$的大小设为$l+\kappa$比特,选择$p$的长度为$2(l+\kappa+\log T)$比特,其中$T$是最大无危胁集$A$的数目,其余则按照上述相同协议发送输入。

客户发送的消息中的每个$y_j$应该是$T$个伪随机值的总和以及要共享的实际秘密。通过选择$p$的大小,如果$y_j$被正确地构造,则这个总和将不涉及任何约简模。因此,我们可以要求每个yj最多只有一个$\kappa+l+\log T$位数,否则拒绝输入。即使$y_j$没有正确构造,这也保证了我们最终得到的份额将是形式$y_j-\sum_{A}F_{K_A}(j)$,因此必须在数值上比$p$小,实际上它必须在区间$[-2^{\kappa+l+\log T},2^{\kappa+l+log T}]$。我们可以很容易地看到,一旦我们知道我们使用的数字有这样的约束,我们可以使用稍后显示的比较协议,实际上唯一的假设是要比较的数字足够小于$p$。因此,服务器可以检查输入数字是否为正数,并根据需要增加或减少。
最后,所使用的公钥加密必须选择安全的密文以对付恶意输入客户端,并且每个加密的明文都必须包含预期接收者的标识。
按照此处所述更改协议会使我们增加$p$的大小,这意味着普遍损失效率,增加数据大小以及检查出价形式的额外工作。另一方面,为了真正作弊,投标人必须编写他自己的客户端程序,并说服服务器端正常客户端仍在使用。就我们的具体情况而言,我们估计出价人以这种方式作弊的风险太小,不足以激发保护它的额外成本。
顺便说一下,我们注意到,可以证明,发送不增加或减少的出价不会增加竞标者的优势,因此无论如何这都是一个小问题。

\subsection{加入安全比较功能}
接下来需要给出如何安全地比较数字。我们将基于对$\mathbb{F}'$的访问来完成这件事。这样结合上一节的结果以及UC合成定理我们就得到了想要的$\mathbb{F}$的实现。注意我们将要比较的数字被假设为具有至多$l$位,而用于秘密共享的素数为$l+\kappa$位长。在如下的协议描述中,我们将对象的算术写作$[d]$。在这个协议中,我们假设可以访问$\mathbb{F}'$,这被理解位一个被$\mathbb{F}'$持有的存有$d$寄存器。在实际的视线中,$[d]$将会是一个$d$的秘密共享。

我们定义一个作用在比特对上的运算$\diamond$为
$$
\begin{pmatrix}
x\\X
\end{pmatrix}
\diamond
\begin{pmatrix}
y\\Y
\end{pmatrix}
=
\begin{pmatrix}
x\wedge y\\
x\wedge(X\oplus Y)\oplus X
\end{pmatrix},
$$
其中$\wedge$表示布尔取和运算符。注意如果我们持有$[a],[b]$,其中$a,b\in \{0,1\}$,那么$[a\oplus b]$可以通过在$\mathbb{F}'$上计算$[a]+[b]-2[ab]$来完成。所以我们可以假设二元取值上的$\oplus$是可用的,从而$\diamond$运算也是可以实现的。容易验证$\diamond$符合结合律。
\newline
\newline
\framebox{
	\begin{minipage}{\textwidth}
	{\heiti 比较协议:}\\
	输入:$[d],[s]$\\
	输出:如果$d\geq s$输出1,否则输出0
	\begin{enumerate}
		\item 对于$i=0,\dots,l+\kappa+1$,调用RandomBit生成$[r_i]$,其中$r_i\in\{0,1\}$是二元随机数。计算$[r]=\sum_i2^i[r_i]$。
		\item 
		计算$[a]=2^{l+\kappa+1}-[r]+2^l+[d]-[s]$。公开
	\end{enumerate}
	\end{minipage}
}
\end{appendix}

%% 个人简历
\begin{resume}

  \resumeitem{个人简历}

  xxxx 年 xx 月 xx 日出生于 xx 省 xx 县。

  xxxx 年 9 月考入 xx 大学 xx 系 xx 专业,xxxx 年 7 月本科毕业并获得 xx 学士学位。

  xxxx 年 9 月免试进入 xx 大学 xx 系攻读 xx 学位至今。

  \researchitem{发表的学术论文} % 发表的和录用的合在一起

  % 1. 已经刊载的学术论文(本人是第一作者,或者导师为第一作者本人是第二作者)
  \begin{publications}
    \item Yang Y, Ren T L, Zhang L T, et al. Miniature microphone with silicon-
      based ferroelectric thin films. Integrated Ferroelectrics, 2003,
      52:229-235. (SCI 收录, 检索号:758FZ.)
    \item 杨轶, 张宁欣, 任天令, 等. 硅基铁电微声学器件中薄膜残余应力的研究. 中国机
      械工程, 2005, 16(14):1289-1291. (EI 收录, 检索号:0534931 2907.)
    \item 杨轶, 张宁欣, 任天令, 等. 集成铁电器件中的关键工艺研究. 仪器仪表学报,
      2003, 24(S4):192-193. (EI 源刊.)
  \end{publications}

  % 2. 尚未刊载,但已经接到正式录用函的学术论文(本人为第一作者,或者
  %    导师为第一作者本人是第二作者)。
  \begin{publications}[before=\publicationskip,after=\publicationskip]
    \item Yang Y, Ren T L, Zhu Y P, et al. PMUTs for handwriting recognition. In
      press. (已被 Integrated Ferroelectrics 录用. SCI 源刊.)
  \end{publications}

  % 3. 其他学术论文。可列出除上述两种情况以外的其他学术论文,但必须是
  %    已经刊载或者收到正式录用函的论文。
  \begin{publications}
    \item Wu X M, Yang Y, Cai J, et al. Measurements of ferroelectric MEMS
      microphones. Integrated Ferroelectrics, 2005, 69:417-429. (SCI 收录, 检索号
      :896KM)
    \item 贾泽, 杨轶, 陈兢, 等. 用于压电和电容微麦克风的体硅腐蚀相关研究. 压电与声
      光, 2006, 28(1):117-119. (EI 收录, 检索号:06129773469)
    \item 伍晓明, 杨轶, 张宁欣, 等. 基于MEMS技术的集成铁电硅微麦克风. 中国集成电路,
      2003, 53:59-61.
  \end{publications}

  \researchitem{研究成果} % 有就写,没有就删除
  \begin{achievements}
    \item 任天令, 杨轶, 朱一平, 等. 硅基铁电微声学传感器畴极化区域控制和电极连接的
      方法: 中国, CN1602118A. (中国专利公开号)
    \item Ren T L, Yang Y, Zhu Y P, et al. Piezoelectric micro acoustic sensor
      based on ferroelectric materials: USA, No.11/215, 102. (美国发明专利申请号)
  \end{achievements}

\end{resume}


%% 本科生进行格式审查是需要下面这个表格,答辩可能不需要。选择性留下。
% 综合论文训练记录表
\includepdf[pages=-]{scan-record.pdf}
\end{document}
